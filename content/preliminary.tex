\chapter{Preliminary Measurements}
\label{sec:prem}
\section{Working Voltage of the Photomultipliers}
As a first step, the optimal working voltages for the photomultipliers (PMTs) have to be found.
Two PMTs were chosen for this measurement as a sample from the whole detector, \texttt{R10B} and \texttt{R20A}.
It is helpful to choose PMTs from different planes to make sure the measured signal is due to a 
particle crossing the detector and not originating in noise.
A schematic for the measuring circuit is shown in \autoref{fig:hvscematic}.\\
\begin{figure}
    \centering 
    \includegraphics[width=0.8\textwidth]{figures/hv.jpg}
    \caption{Schematic representation of the coincidence counting circuit for the PMTs. 
    The signal propagation is from left to right starting with the high voltage sources of the PMTs and ending with the scaler. In our setup the signals of \texttt{R10B} and \texttt{R20A} were used.}
    \label{fig:hvscematic}
\end{figure}

\vskip 0.09 cm
Only signals measured with both PMTs are taken into account. To determine these coincidences 
logic signals are needed and for this, discriminators with a threshold of $\SI{40}{mV}$ are used. 
They convert the analogous signals of each PMTs into NIM standard signals. Using a logic unit module, we get
the coincidences from the logic \texttt{AND} of both PMT signals. The NIM signals 
last for $\SI{40}{\ns}$ each so the signals do not have to arrive at the same time but within 
$\SI{40}{\ns}$ of each other because the muons arrive in the different planes at different times 
and not necessarily at the same distance from the PMTs.\\
\vskip 0.09 cm
For the measurement, the \texttt{R10B} PMT works at a fixed high voltage of $\SI{775}{V}$ while the high voltage of 
\texttt{R20A} is varied from $\SI{700}{V}$ to $\SI{900}{V}$. Each measurement lasts $\SI{200}{s}$.
The taken data is shown in \autoref{fig:hv}. The plateau is between $\SI{862}{V}$ 
and $\SI{887}{V}$ with a
middle value of $\SI{875}{V}$. This value is used as a fixed value for the \texttt{R20A} PMT when the voltage of the \texttt{R10B}
PMT is varied in the same range of high voltages. The plateau for this PMT is between $\SI{800}{V}$ and $\SI{850}{V}$
and the middle value is $\SI{825}{V}$.
Both measurements were also stopped when the PMTs are operating in the discharge region due to the high voltage to avoid damaging the PMTs.\\
\vskip 0.09 cm
The results are shown in \autoref{fig:hvPMT1} and \autoref{fig:hvPMT2}. All uncertainties are obtained 
assuming a Poissonian distribution, meaning $\sigma \propto \sqrt{N}$, where $N$ is the number of
coincidences. Because the output of the high voltage sources can fluctuate, the central values of the plateaus, $\SI{875}{V}$ for the \texttt{R20A} PMT and 
$\SI{825}{V}$ for the \texttt{R10B} PMT, can be used as the optimal working voltages to ensure
a voltage output in the region of the plateau.
\vskip 0.09 cm
\begin{figure}
        \centering
        \begin{subfigure}[b]{0.48\textwidth}
        \includegraphics[width=\textwidth]{plots/hvR20A.pdf}
        \captionof{figure}{Data of the \texttt{R20A} PMT.
        The central value of the plateau is $\SI{875}{V}$.}
        \label{fig:hvPMT1}
    \end{subfigure}\hfill
\begin{subfigure}[b]{0.48\textwidth}
        \includegraphics[width=\textwidth]{plots/hvR10B.pdf}
        \captionof{figure}{Data of the \texttt{R10B} PMT.
        The central value of the plateau is $\SI{825}{V}$.}
        \label{fig:hvPMT2}
\end{subfigure}
\caption{Measured number of coincidences as function of high voltages
of the \texttt{R10B} PMT and the \texttt{R20A} PMT.
The dashed line indicates the central values of each plateau. The striped red areas mark the found plateaus.}
\label{fig:hv}
\end{figure}


\section{Threshold curves}
After the optimal working voltages was found, the determination of the optimal threshold voltage of the chosen PMTs started.
A good determination of the threshold is necessary to suppress noise on the one hand, but also to make sure that real signals are not
filtered as well on the other hand. So the threshold can neither be too low nor too high.\\

Similar to the step before, a counting experiment of coincidences concerning the threshold voltages is conducted. This time the circuit arrangement is shown in 
\autoref{fig:threshold_scematic}.
\begin{figure}
   \centering
   \includegraphics[width=0.87\textwidth]{figures/thresh.jpg}
   \caption{Schematic representation of the electronic circuit used to find the optimal threshold voltage.
   The signal flows from left to right, starting with the PMTs and Ending with the scaler.}
   \label{fig:threshold_scematic}
\end{figure}

In principle, the threshold of the discriminator of one plane is kept fixed at $\SI{15}{mV}$, while 
the threshold of the other discriminator is being varied from $\SI{5}{mV}$ to $\SI{250}{mV}$ in irregular steps. 
Using a discriminator and a Fan-in-Fan-out system as shown in \autoref{fig:threshold_scematic}, it is possible to vary
the threshold of the discriminators of both PMTs at the same time. Each counting measurement was conducted in $\SI{60}{s}$.\\ 
The results for the \texttt{R10} and \texttt{R20} planes can be seen in \autoref{fig:threshR10R20}.
Results from other planes have to be taken from \autoref{sec:appendix}, figures \autoref{fig:appthresh1} to \autoref{fig:appthresh6}.\\
\begin{table}
    \centering
    \caption{Overview of optimal threshold voltages for each plane.}
    \label{tab:thresh}
    \begin{tabular}{c c | c c | c c}
    \toprule
      {Plane} & {Voltage / mV} & {Plane} & {Voltage / mV} & {Plane} & {Voltage / mV} \\
    \midrule
    R00          & 50 & R20          & 55 & L01  & 55\\
    R10          & 70 & R21          & 55 & L11  & 75\\
    \textit{R10} & \textit{45} & L00          & 30 & L20  & 50\\
    R01          & 55 & L10          & 35 & L21  & 50\\
    R11          & 60 & \textit{L10} & \textit{60} & & \\
    \bottomrule
    \end{tabular}
    \end{table}
The data shows a higher number of coincidences at low threshold voltages, which decreases with rising voltages.
This behavior is the same for all PMTs although not all curves show it that clearly.  
Also, a plateau of optimal threshold voltages can be seen. From the middle of this plateau, a threshold voltage 
has to be chosen as the optimal threshold value. As before the uncertainties will be taken 
into consideration using the Poissonian error. The optimal threshold voltages of all PMTs are shown in 
\autoref{tab:thresh}.
\begin{figure}
    \centering
    \begin{subfigure}[b]{0.48\textwidth}
    \includegraphics[width=\textwidth]{plots/threshR20.pdf}
    \captionof{figure}{\texttt{R20}}
\end{subfigure}\hfill
\begin{subfigure}[b]{0.48\textwidth}
    \includegraphics[width=\textwidth]{plots/threshR10.pdf}
    \captionof{figure}{\texttt{R10}.}
\end{subfigure}
\caption{Measured number of coincidences concerning varying threshold voltages
of the \texttt{R10} PMTs and the \texttt{R20} PMTs.
The dashed line indicates the central values of each plateau. The striped orange areas mark the found plateaus. The central value of the 
\texttt{R20} PMTs is $\SI{55}{mV}$ and of the \texttt{R10} PMTs it is $\SI{70}{mV}$.}
\label{fig:threshR10R20}
\end{figure}
\section{Delay Curve}
With systematical variation of the delay between the PMTs \texttt{R10} and \texttt{R20} a delay 
curve for these PMTs can be created. This delay curve is shown in \autoref{fig:delay}.
For this measurement delay is added to the signals and the coincidences are counted using scalers in 
$60 \; \symup{s}$. \\
The delay curve describes the number of coincidences measured for various delays between the PMTs.
A negative delay time indicates a delay for the signal of \texttt{R10}, while a positive delay 
represents a delay of the signal of \texttt{R20}. The flanks of the curve can be fitted with linear 
functions of the form 
\begin{equation*}
    N(\Delta t) = a \cdot \Delta t + b
\end{equation*}
with $N$ as the number of coincidences and $\Delta t$ as the delay time. 
The parameter are 
\begin{align*}
    a_{\text{left}} = (64.9 \pm 7.6)\; \symup{\frac{1}{ns}} \qquad
    b_{\text{left}} = 2328.6 \pm 251.4
\end{align*} 
for the left flank from $-36\;\symup{ns}$ to $-25\;\symup{ns}$ and    
\begin{align*}
    a_{\text{right}} = (-49.4 \pm 6.3)\; \symup{\frac{1}{ns}} \qquad
    b_{\text{right}} = 1972.6 \pm 227.3
\end{align*} 
for the right flank in the interval $32\;\symup{ns}$ to $40\;\symup{ns}$.
The plateau between the two flanks has an average value of $343.3 \pm 18.6$. With the estimated 
parameters and the plateau value the time values of the half maximums are 
\begin{equation*}
    \Delta t_{\text{left}} = -33 \pm 5 \qquad
    \Delta t_{\text{right}} = 36 \pm 7.
\end{equation*}
Thus, the full width half maximum can be estimated as 
\begin{equation*}
    \Delta t_{\text{FWHM}} = 70 \pm 9
\end{equation*}
which is a measure for the resolution of the PMTs.
\begin{figure}
    \centering
    \includegraphics[width=0.6\textwidth]{plots/delay.pdf}
    \caption{Delay Curve between the PMTs \texttt{R10} and \texttt{R20}. The number of coincidences is plotted against the delay between the the PMTs.
    Furthermore, linear fits of the flanks and the FWHM are plotted.}
    \label{fig:delay}
\end{figure}
\section{TDC callibration}
To convert the TDC time units into real time units the various TDC channels have to be calibrated.
In this step malfunctioning TDC channels can be detected as well. To accomplish this,
a fake trigger signal is needed at a rate of about $4 \; \symup{Hz}$ or more. The fake signal is generated using the 
end marker of a dual timer. First, a signal is send to the START of the TDCs. Then this signal is used 
to create a delayed signal which stops the various channels. The time difference between start and stop signal was measuered with 
an oscilloscope. \\
For all channels and for each time difference ranging from $0.5 \; \symup{\mu s}$ to $4.8 \; \symup{\mu s}$ 
at least 200 measurements have been done.
Special attention had to be given to the TDC channels which generated an overflow (i.e. a value greater than 4095)
and other anomilies.
These channels could not be fully in the data taking since they would provide unreliable
and incomplete data.\\
For all channels a linear function of the form 
\begin{equation*}
    \text{TDC Time} = a \cdot t + b
\end{equation*}
is fitted to the data, where $t$ represents the real time values.
These are shown for all channels in the figures \ref{fig:tdc01} to
\ref{fig:tdc1415} in \autoref{sec:tdc_figures}.
The estimated parameters are shown in \autoref{tab:tdc}. Furthermore, a $\chi^2$-test is conducted 
for all fits, although it has to be taken into account that for some fits a cut-off due to overflow of
the time values had to be done. The results of the reduced $\chi^2$-test and the cut-off values 
are also shown in the table.\\
The low values resulting from the tests indicate a good fit of the model to the data, while still being affected by the uncertainties 
of the measurements. The higher the value is, the lower the model fits the data.
\begin{table}[!htp]
    \centering
    \caption{Parameters of the fits for the TDC calibration with cut-off values and $\chi^2$ values of the fits.}
    \label{tab:tdc}
    \begin{tabular}{c | c c c | c}
    \toprule
    {Channel} & {Slope $a$ / $\text{TDC Units} / \symup{ns}$} & {Offset $b$ /$\symup{ns}$} & {cut-off / $\symup{\mu s}$} & {$\chi^2/\symup{d.o.f.}$} \\
    \midrule
    TDC 1 & & & \\
    1 & 0.754 \pm{} 0.001 & 10 \pm{} 5& - & 0.037 \\
    2 & 0.766 \pm{} 0.001 & 6  \pm{} 5& -  & 0.038 \\
    3 & 0.747 \pm{} 0.001 & 10 \pm{} 5& - & 0.037 \\
    4 & 0.758 \pm{} 0.001 & 10 \pm{} 5& - & 0.037 \\
    5 & 0.738 \pm{} 0.001 & 11 \pm{} 4& - & 0.036 \\
    6 & 0.719 \pm{} 0.001 & 10 \pm{} 4& - & 0.035 \\
    7 & 0.732 \pm{} 0.001 & 12 \pm{} 4& - & 0.034 \\
    8 & 0.742 \pm{} 0.001 & 12 \pm{} 4& - & 0.035 \\
    & & & &\\ \hline
    TDC 2 & & & & \\
    1 & 0.808 \pm{} 0.001 & 44 \pm{} 6 & 4.6& 0.053 \\
    2 & 0.943 \pm{} 0.005 & 48 \pm{} 10& 3.5 & 0.120 \\
    3 & 0.802 \pm{} 0.002 & 37 \pm{} 6 & 4.6&  0.056 \\
    4 & 0.791 \pm{} 0.002 & 43 \pm{} 5 & 4.7& 0.048 \\
    5 & 0.783 \pm{} 0.001 & 15 \pm{} 5 & -& 0.042 \\
    6 & 0.786 \pm{} 0.001 & 29 \pm{} 5 & 4.7& 0.042 \\
    7 & 0.813 \pm{} 0.002 & 44 \pm{} 6 & 4.5& 0.062 \\
    8 & 0.802 \pm{} 0.002 & 39 \pm{} 6 & 4.6& 0.057 \\
    \bottomrule
    \end{tabular}
    \end{table}
With the cut-off of overflow values the $\chi^2$-test values of the fits of all channels are 
small with the biggest value being produced in channel 2 of TDC 2, which will not be used due to 
its cut-off anyway. Thus, the fits can be consired succesfull. 
For the data aquisition the channels 3, 4, 5 and 6 of both TDCs have been chosen.
In the data analysis both parameter will be used to convert the TDC times into real times although the 
offset $b$ being small in comparison to the common TDC times ranging from 0 to 4095.


