\chapter{Final Results and Conclusions}
\label{sec:results}
In this report a basic 
trigger system for a detector composed of three scintilator planes and two iron slabs 
was used to detect stopping cosmic muons and their lifetime. 
From the total of 1184834 measured trigger events, only 6505 resulting in an efficiency value of 
\begin{equation*}
    \epsilon = \frac{6505}{1184834} \approx 5 \cdot 10^{-3}.
\end{equation*}
This efficiency is far from optimal but should be expected given the number of possible fake signals
and other background signals. Using the fit on the available data, a mean muon lifetime could be 
estimated as 
\begin{equation*}
    \tau = (2.1 \pm 0.1) \; \symup{\mu s}.
\end{equation*}
This calculated value is in $4.4\%$ agreement with the literature value of $\tau_{\text{pdg}} = 2.1969811 \pm 0.0000022$ \cite{pdg}.
Since the relative deviation is smaller and the true value well inside the 
standard deviation and, therefore, is consistent with it, this measurement will be considered successfull.\\
Furthermore, an attempt to calculate the lifetime of bound muons in iron was made as well. 
The analyisis results in a lifetime of 
\begin{equation*}
    \tau_{\text{bound}} = (0.16 \pm 0.04) \; \symup{\mu s},
\end{equation*}
which deviates $22.3\%$ of the experimental value of $\tau_{\text{bound}} = 0.206 \;\symup{\mu s}$ \cite{lvd}.
Since the effect of the decay of captured muons is much smaller than the effect of free 
muons, more measured events would be needed to improve this measurement to eventually
become valid and consistent.
