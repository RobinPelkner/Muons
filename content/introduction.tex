\chapter{Introduction}
This experiment aims to determine the lifetime of muons. Muons are produced by cosmic ray showers in the higher atmosphere.
Incoming Protons interact with the atmosphere and produce pions, which can decay into muons that are reaching the surface of the earth. \\
In this experimental setup, the incoming muons are slowed down and stopped by layers of iron and then detected with plastic scintillators.\\
This report is structured as follows: in this chapter, an introduction to the physical aspects of muons is given.
\hyperref[sec:setup]{Chapter~\ref*{sec:setup}} describes the used components and the experimental setup. In \hyperref[sec:prem]{Chapter~\ref*{sec:prem}}
the preliminary measurements for the calibration of the setup are described and their results are discussed. \hyperref[sec:trigger]{Chapter~\ref*{sec:trigger}} discusses the composition of the experiment and the process of data taking, while \hyperref[sec:data_analysis]{Chapter~\ref*{sec:data_analysis}} presents the data analysis. \hyperref[sec:results]{Chapter~\ref*{sec:results}} concludes this report.


\newpage
\section{General introduction to muons}
The standard model of elementary particles differentiates between quarks, leptons, and bosons.
Quarks, as well as leptons, can be divided into three generations.
The muons and the antimuon respectively are the second generation of leptons together with their associated neutrinos.
Like the electron, muons have an spin $1/2$ and a negative elementary charge (a positive elementary charge for the antimuon).
Although the similarities, they differ concerning their masses and their lifetimes.\\
While the electron is a stable particle, the muon has a finite lifetime, which this experiment is supposed to measure. 
The lifetime, given as an average result of past measurements, is $\tau = \SI{2.1969811 \pm 0.0000022}{\micro\second}$.
Their mass is $m_{\mu} = \SI{105.6583745 \pm 0.0000024}{\mega\eV}$, which is 206 times the mass of the electron \cite{pdg}.

\section{Muons from cosmic ray showers} 
Muons are naturally produced in cosmic ray showers. 
The cosmic ray is mostly protons, but also electrons and atomic nuclei.
These particles originate in various galactic and intergalactic sources like the sun, the Milky Way, and other galaxies.
When a cosmic particle (e.g; Proton) strikes and interact the atmosphere, it produces a hadronic shower.
The protons hadronize mostly into $\pi$ mesons, because they are the lightest mesons, but also into $K$ mesons and other hadrons.
The primary cosmic rays are completely hadronized at altitudes about $\SI{20}{km}$. 
Because $\pi$ mesons and $K$ mesons are not stable with lifetimes in the order of magnitude of $\SI{e{-8}}{\second}$ they decay very fast. 
For the decay into muons, the $\pi$ mesons have a branching ratio of $\Gamma = (99.98770\pm{0.00004})\%$, while the $K$ mesons have a branching ratio of $\Gamma = (63.56\pm{0.11})\%$ \cite{pdg}.
It should be noted that these decays into muons only occur with charged $\pi$ and $K$ mesons.
Neutral mesons are also produced in these showers, but they do not decay into charged muons.
The decay of the $\pi$ mesons takes place via the weak interaction 
\begin{align*}
    \pi^{+} &\to \mu^{+} + \nu_{\mu} \\
    \pi^{-} &\to \mu^{-} + \bar{\nu}_{\mu}.
\end{align*}
This process is analogous to $K$ mesons.
The muons also only decay via the weak interaction as follows 
\begin{align*}
    \mu^{-} &\to e^{-} + \bar{\nu}_{e} + {\nu}_{\mu} \\
    \mu^{+} &\to e^{+} + \nu_{e} + \bar{\nu}_{\mu}.
\end{align*}
Because the muons have a longer lifetime than pions. Hence, muons can reach the surface of the earth and can be detected through their decay. 
A schematic with typical processes, which lead to the production of muons, can be seen in \autoref{fig:cosmic_ray_showers}.
\begin{figure}
    \centering
    \includegraphics[width=0.5\textwidth]{figures/cascade.png}
    \caption{Typical shower processes after an incoming cosmic ray. The primary rays induce secondary particles by scattering with the 
    molecules of the atmosphere. These secondaries can decay or scatter again with the atmosphere and produce more particles in the process. 
    This is called a decay chain or a cosmic ray shower\cite{nasa}.}
    \label{fig:cosmic_ray_showers}
\end{figure}
\\
Suppose a factor $\beta = 0.999c$ and a muon lifetime $\tau = \SI{2e-6}{s}$, classically this is not enough.
But since the beta factor is high enough we end up in the relativistic regime and we can calculate
\begin{equation*}
\tau '  = \frac {1}{\sqrt{1- \beta ^2}} \tau = 1.4 \cdot 10^{-4}
\end{equation*}
which is a lot longer than the classical result would allow. 
Once they arrive on the surface of earth, we can use experimental apparatus to detect them.
The basic idea is that when a particle passes through matter they release part of its energy which is converted into new particles or radiation.
The amount of released energy is described by the
Bethe Bloch Formula:\\\\
\begin{equation*}
-\frac {\symup{d}E}{\symup{d}x} = 2 \pi N_a r_{e}^{2}m_{e}c^{2}\rho \frac{Z}{A} \frac {z^{2}}{\beta^{2}} [ ln(\frac{2m_{e} \gamma^{2} v^{2} W_{max}}{I^{2}}  - 2 \beta^{2} - \delta 2 \frac {C}{Z})]
\end{equation*}
In general, a heavy, charged particle passing through matter loses energy through five mechanisms
\begin{enumerate}
    \item Inelastic collision from atomic electron
    \item Elastic scattering from nuclei
    \item Cherenkov radiation
    \item Nuclear reaction
    \item Bremsstrahlung.
\end{enumerate}
The last three are usually not included in the calculation because the 
cross-section of their reaction is really low. The transferred energy from the particle 
to the atom will cause ionization or excitation of them and we can construct an apparatus
with specific material to take advantage of this process. In this case, a scintillator
coupled with a photomultiplier is used.\\
A Scintillator is a device constructed with particular materials that exhibit a propriety 
called \textit{luminescence} which consists of the ability to convert a certain form of energy
in visible light. The characteristics of a good scintillator are
\begin{enumerate}
   \item high efficiency for conversion of energy into fluorescent radiation
    \item transparency to its fluorescent radiation so it can permit the transmission of the produced light
    \item short decay constant
    \item emission in a spectral range consistent with the spectral response of existing photomultipliers. 
\end{enumerate}
So thanks to this we can translate an unknown particle into something known. 
Since we want a signal out of this we can use the radiation emitted to create a current 
employing the photoelectric effect.\\ 
Then once these photoelectrons are produced they can create a current which usually is very
weak and we want a device that can amplify it using a photomultiplier.\\
A photomultiplier consists of a cathode made of photosensitive material followed by an
electron collection system, an electron multiplier section (or dynode string as it is usually called)
and finally, an anode from which the final signal can be taken.
When photons hit the photocathode electrons are released which are accelerated towards
the next dynode causing more electrons to be released and so on till they reach the anode
where the signal is produced. The photocathode is a device that converts incident light
into a current of electrons by the photoelectric effect. To facilitate the passage of this light,
the photosensitive material is deposited in a thin layer on the inside of the photomultiplier
window which is usually made of glass or quartz.
The process is modeled by the Einstein formula\\\\
\begin{equation*}
E=hv - \phi 
\end{equation*}
where $E$ is the kinetic energy of emitted electrons, $v$ is the frequency of incident light
and $\phi$ the work function. It is clear that a certain minimum frequency
is required before the photoelectric effect may take place.
